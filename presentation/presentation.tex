\documentclass{beamer}
\usetheme{Copenhagen}

\usepackage{../quiver}
\usepackage{../math}
\usepackage{eucal}

%Information to be included in the title page:
\title{Localizations of models of dependent type theory}
\author{Author: Matteo Durante \quad\quad\quad Advisor: Hoang-Kim Nguyen}
\institute{Regensburg University}

% Presentation-only operators (not in shared math.sty)

\begin{document}

\theoremstyle{plain}

\newtheorem{thm}{Theorem}[section]
\newtheorem{prop}{Proposition}[section]
\newtheorem{defn}{Definition}[section]
\newtheorem{conj}{Conjecture}[section]
\newtheorem{lem}{Lemma}[section]
\newtheorem{construction}{Construction}[section]

\frame{\titlepage}

\begin{frame}
  \frametitle{Objective}

  A modern proof of the following theorem.

  \begin{thm}[Kapulkin 2015]
    Given a dependent type theory $\mathbf{T}$ with $\Sigma$-, $\Ids$- and
    $\Pi$-types, the $\infty$-localization of its syntactic category $\Syn{T}$ is a
    locally cartesian closed $\infty$-category.
  \end{thm}
\end{frame}

\begin{frame}
  \frametitle{Dependent type theory}
  
  \begin{block}{What}
    A theory of computations and a foundation of mathematics.
  \end{block}
  \pause

  \begin{block}{Objects}
    \emph{Dependent types $A$} and their \emph{terms $x:A$} in \emph{contexts
    $\Gamma=(x_0:A_0,\ldots,x_n:A_n)$}.
  \end{block}
  \pause

  \begin{block}{Structural rules}
    How to work with \emph{variables}.
  \end{block}
  \pause

  \begin{block}{Logical rules}
    Construct new types and their terms from old, carry out
    computations. They provide \emph{$\Sigma$-types}
    $\Sigma(A,B)$, \emph{$\Pi$-types}
    $\Pi(A,B)$, \emph{$\Ids$-types} $\Ids_A$, \emph{natural-numbers-type}
    $\Nat$\ldots
  \end{block}
\end{frame}

\begin{frame}
  \frametitle{Models}

  \begin{block}{Idea}
    To reason about a theory we can look at its interpretations.
  \end{block}
  \pause

  \begin{block}{Problem}
    Providing a model of dependent type theory is hard.
  \end{block}
  \pause

  \begin{block}{Solution}
    Defining a class of algebraic models.
  \end{block}
\end{frame}

\begin{frame}
  \frametitle{Modeling structural rules}

  \begin{definition}[contextual categories]
    A category $\sfC$ with:
    \begin{enumerate}
      \item a grading on objects (or \emph{contexts}) $\Ob\sfC=\coprod_{n\in\bbN}\Ob_n\sfC$;
      \item a unique and terminal object in $\Ob_0\sfC$, the \emph{empty context};
      \item a map $\cft_n\colon\Ob_{n+1}\sfC\rightarrow\Ob_n\sfC$ for each
        $n\in\bbN$;
      \item \emph{basic dependent projections}
        $p_A\colon\Gamma.A\rightarrow\cft_n(\Gamma.A)=\Gamma$;
      \item a functorial choice of pullback squares
        \[\begin{tikzcd}[ampersand replacement=\&]
          {\Delta.f^*A} \& {\Gamma.A} \\
          \Delta \& \Gamma
          \arrow["{p_A}", from=1-2, to=2-2]
          \arrow["{p_{f^*A}}"', from=1-1, to=2-1]
          \arrow["{q(f,A)}", from=1-1, to=1-2]
          \arrow["f"', from=2-1, to=2-2]
        \end{tikzcd}.\]
    \end{enumerate}
  \end{definition}
\end{frame}

\begin{frame}
  \frametitle{Modeling logical rules}

  \begin{block}{Extra structure}
    $\Ids$-types require from $\Gamma.A$ an $\Ids$-object $\Gamma.A.A.\Ids_A$...
    
    $\Pi$-types require from $\Gamma.A.B$ a $\Pi$-object $\Gamma.\Pi(A,B)$, an
    evaluation map $\app_{A,B}\colon\Gamma.\Pi(A,B).A\rightarrow\Gamma.A.B$,
    $(f,a)\mapsto(a,\app(f,a))$...
  \end{block}
  \pause

  \begin{example}
    If $\mathbf{T}$ has some logical rules, then its (contextual) syntactic
    category $\Syn{T}$ has the corresponding logical structures. It is freely
    generated by the theory: objects are contexts, morphisms
    $[x_0:A_0,\ldots,x_n:A_n]\rightarrow[y_0:B_0,\ldots,y_m:B_m]$ are
    tuples of terms $(f_0:B_0,\ldots,f_m:B_m)$ derivable from
    $x_0:A_0,\ldots,x_n:A_n$.
  \end{example}
\end{frame}

\begin{frame}
  \frametitle{Bi-invertibility}

  \begin{defn}[bi-invertible map]
    A map $f\colon\Gamma\rightarrow\Delta$ in a contextual category with
    $\Ids$-structure $\sfC$ for which we can provide:
    \begin{enumerate}
      \item maps $g_1\colon\Delta\rightarrow\Gamma$,
        $\eta\colon\Gamma\rightarrow\Gamma.(1_\Gamma,g_1\cdot f)^*\Ids_\Gamma$;
      \item maps $g_2\colon\Delta\rightarrow\Gamma$,
        $\epsilon\colon\Delta\rightarrow\Delta.(1_\Delta,f\cdot g_2)^*\Ids_\Delta$.
    \end{enumerate}
  \end{defn}
  \pause

  \begin{block}{Question}
    What if we localize at bi-invertible maps?
  \end{block}
\end{frame}

\begin{frame}
  \frametitle{Fibrational structure}
  
  \begin{defn}[$\infty$-categories with weak equivalences and fibrations]
    A triple $(\cC,W,Fib)$ where:

    ...a weakening of the definition of fibration categories, with $\cC$ an
    $\infty$-category.
  \end{defn}
  \pause

  \begin{thm}[Avigad-Kapulkin-Lumsdaine 2013]
    A contextual category with $\Sigma$- and $\Ids$-structures defines a
    fibration category, where weak equivalences are bi-invertible maps and
    fibrations are maps isomorphic to  compositions of basic dependent
    projections $p_A\colon\Gamma.A\rightarrow\Gamma$.
  \end{thm}
\end{frame}

\begin{frame}
  \frametitle{Localizing fibrational categories}

  \begin{construction}[fibrant slice $\cC(x)$]
    Given a fibrant object $x$ in $\cC$, lift the fibrational structure through
    $\cC/x\rightarrow\cC$ and then take the subcategory of fibrant objects
    of $\cC/x$.
  \end{construction}
  \pause

  \begin{prop}[Cisinski]
    Given an $\infty$-category with weak equivalences and fibrations $\cC$, if
    for every fibration $f\colon x\rightarrow y$ between fibrant objects the
    pullback functor between fibrant slices $f^*\colon\cC(y)\rightarrow\cC(x)$
    has a right adjoint preserving trivial fibrations, then $L(\cC)$ is locally
    cartesian closed.
  \end{prop}

\end{frame}

\begin{frame}
  \frametitle{Localizations of models are cartesian closed}

  \begin{thm}[Kapulkin 2015]
    Given a dependent type theory $\mathbf{T}$ with $\Sigma$-, $\Ids$- and
    $\Pi$-types, the localization of its syntactic category $\Syn{T}$ is a
    locally cartesian closed $\infty$-category.
  \end{thm}
  \pause

  \begin{proof}
    For any basic dependent projection $p_A\colon\Gamma.A\rightarrow\Gamma$,
    there exists a right adjoint to
    $p_A^*\colon\Syn{T}(\Gamma)\rightarrow\Syn{T}(\Gamma.A)$ given by
    \[(p_A)_*(\Gamma.A.\Theta)=\Gamma.\Pi(A,\Theta)\]
    with counit induced by $\app_{A,\Theta}$. It preserves the fibrational
    structure.
  \end{proof}
\end{frame}

\begin{frame}
  {\centering Thank you for your attention!\par}
\end{frame}

\begin{frame}
  \frametitle{Issues}

  Essentially, folklore.

  \begin{block}{Extended structures}
  We used extensions of the $\Ids$-, $\Sigma$- and $\Pi$-structures: from
  $\Gamma.A.\Theta$ we have a $\Pi$-object $\Gamma.\Pi(A,\Theta)$ only when
  $\cft(\Gamma.A.\Theta)=\Gamma.A$, however $\Theta$ represents an arbitrary
  extension, like $\Gamma.A.B$ or $\Gamma.A.B.C$.
  These extended structures
  were mentioned in the literature, but not entirely defined.
  \end{block}

  \begin{block}{Internal languages}
  Researchers often argue by relying on them, intuitive but undefined tools. We
  chose to work with syntactic categories because then we can reason as we wish.
  \end{block}
\end{frame}

\begin{frame}
  \frametitle{Why is dependent type theory cool?}

  \begin{enumerate}
    \item Closely linked to \emph{computations} and \emph{computer science},
      makes proof assistants possible.
    \item Enough by itself as a foundation, unlike set theory or propositional
      calculus.
    \item \emph{Proofs} are internal objects.
    \item Better treatment of \emph{equality}.
    \item Makes ``fully faithful $+$ essentially surjective = equivalence''
      independent from the axiom of choice.
    \item Homotopical interpretation in $\infty$-groupoids.
  \end{enumerate}
\end{frame}

\begin{frame}
  \frametitle{Internal languages conjecture}

  \begin{conj}[Kapulkin-Lumsdaine 2016]
    The horizontal maps, given by simplicial localization, induce
    equivalences of $\infty$-categories.
    \[\begin{tikzcd}[ampersand replacement=\&]
      {\Cxl_{\Sigma,1,\Ids,\Pi}} \& {\LCCCi} \\
      {\Cxl_{\Sigma,1,\Ids}} \& {\Lexi}
      \arrow[from=1-1, to=2-1]
      \arrow[from=1-2, to=2-2]
      \arrow[from=2-1, to=2-2]
      \arrow[from=1-1, to=1-2]
    \end{tikzcd}\]
  \end{conj}

  A proof by Nguyen-Uemura has recently become available on arxiv.

  One hopes to extend this to an equivalence between $\Cxl_{\HoTT}$ and
  $\ElToposi$.
\end{frame}

\end{document}
